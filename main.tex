\documentclass[a4paper, 11pt]{article}

\oddsidemargin=-1cm
\topmargin=-1.5cm
\textwidth=18cm
\textheight=22cm

\usepackage{graphicx}
\usepackage{fancyhdr}

\pagestyle{fancy}
\lhead{\footnotesize\sc ASHPC24 -- Austrian-Slovenian HPC Meeting 2024}
\chead{}
\rhead{\footnotesize\sc Grundlsee, June 10 -- June 13, 2024} 

\usepackage{enumitem}
\newlist{reflist}{enumerate}{1}
\setlist[reflist,1]{label={[\arabic*]},topsep=0pt,leftmargin=6mm,itemindent=0mm,labelsep=2mm}

\newcommand{\affiliation}[1]{\small{\emph{#1}}}

\renewcommand{\thesection}{}
\renewcommand{\thesubsection}{}
\setlength{\parindent}{0pt}
\setlength{\parskip}{7pt}

\makeatletter
\renewcommand\section{\@startsection {section}{1}{\z@}%
  {-3.5ex \@plus -1ex \@minus -.2ex}%
  {2.3ex \@plus.2ex}%
  {\centering\normalfont\Large\bfseries}}
\makeatother

\makeatletter
\renewcommand\subsection{\@startsection {subsection}{1}{\z@}%
  {-3.5ex \@plus -1ex \@minus -.2ex}%
  {2.3ex \@plus.2ex}%
  {\centering\normalfont\large\bfseries}}
\makeatother

\begin{document}

\section{Live resource management in HPC applications:\\ Resilient hybrid computing in IMS' MBMW processing pipeline}
\subsection{David Pfahler, Harald Höller-Lugmayr}

\begin{center}
\setlength{\parskip}{0pt}
\affiliation{IMS Nanofabrication GmbH, Brunn am Gebirge, Austria}
\setlength{\parskip}{7pt}
\end{center}

% Abstract
% ========
We address the intricate challenge of managing resource limitations and hardware utilization 
within the Multi-Beam Mask Writer (MBMW) processing pipeline~[1].
Our primary objective is to enhance stability and availability by dynamically adapting to varying computational demands during the real-time processing for mask writing with high bandwidth on an HPC cluster.
It's crucial to note the inherent variability of parameters and input data and computational demand on the hardware components. In combination with an industry in constant pursuit of higher complexity~[2], upfront knowledge of expected resource amounts is very challenging. This uncertainty emphasizes the significance of our adaptive resource management approach in addressing the dynamic nature of the MBMW processing pipeline.

Our proposed framework integrates three key techniques.
Firstly, real-time tracking of memory allocations within each process, covering both GPU and CPU usage, provides granular insights into the specific resource demands of individual components.
This enables precise resource management. 
The second technique involves aggregating the tracked memory allocations based on insights from individual processes.
This holistic approach offers a comprehensive view of global resource usage, facilitating the establishment of globally tracked limitations.
This strategy enhances the adaptability of resource management across the entire MBMW processing pipeline, contributing to overall system stability.
With this, offloading tasks to the GPU devices are controlled by memory occupation and allow optimal hardware utilization.  
The third technique focuses on the dynamic employment of down-scaling mechanisms in response to exceeded limitations.
This proactive approach effectively eliminates the risk of Out-of-Memory situations, ensuring continuous operation by adapting to resource limitations in real time.
This adaptive response mechanism significantly contributes to improving the stability of the MBMW processing pipeline.

Looking forward, our approach holds the potential for further enhancement to incorporate additional resource metrics beyond CPU and GPU memory. 
Metrics that may prove relevant for HPC applications include CPU utilization, the number of concurrent threads and processes, the number of open file descriptors, and domain-specific metrics such as the count of input polygons for the MBMW processing pipeline.

Validation efforts encompass comprehensive experiments using real-world scenarios on an HPC cluster.
The results highlight a substantial improvement in the system's ability to cater to changing computational demands.
This enhancement, demonstrated through improved stability and availability, is quantitatively measured by the system's 99~\% up-time.
This metric ensures a reliable assessment of the proposed approach.

% ============================================================== %
\vspace{4ex}
% -------------------------------------------------------------- %
{\bf{References}}
\begin{reflist}
  \item [{[1]}] 
  Christof Klein and Elmar Platzgummer,
  ``MBMW-101: World’s 1st high-throughput multi-beam mask writer'',
  SPIE Photomask Technology 2016,
  \textbf{9985}, 998505
  (2016).

  \item [{[2]}]
  Kyungsup Shin, Harald Höller-Lugmayr et. al.,
  ``New multi-beam mask data preparation method for EUV high volume data'',
  XXIX Symposium on Photomask and Next-Generation Lithography Mask Technology, Proc. SPIE,
  \textbf{12915}, 116-125
  (2023).
\end{reflist}
\end{document}

